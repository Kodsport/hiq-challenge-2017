\problemname{Haiku}
A haiku is a Japanese form of poetry.
It consists of three phrases of 5, 7 and 5 syllables each.

Once a year, HiQ has a haiku contest, where employees submit their best poems.
The best poems are then judged based on a wide range of aspects, such as
\begin{itemize}
    \item creativity
    \item simplicity
    \item beauty
    \item whether the poem is actually a haiku
\end{itemize}
This last point turned out to be quite a challenge for the judges (many problems arose when half the judges indexed syllables starting at 0 and the other half at 1).

Can you help the judges to determine whether a submitted poem is a haiku, given a set of syllables?
Note that there \textbf{may exists multiple decompositions} of a single word in the poem into syllables.
In this case, you should determine whether \textbf{some} decomposition is a haiku.

\section*{Input}
The first line of input contains a single integer $1 \le S \le 100$, the number of syllables.
The next line contains $S$ words separated by spaces (the syllables).
Each syllable contains at most 7 lowercase letters \texttt{a-z}.

Then, three lines containing the poem follow.
Each line contains a (non-empty) list of words separated by spaces.
The length of each line is at most 100 characters (including spaces).

It is guaranteed that there exists \textbf{at least one} decomposition of the poem into the given syllables.

\section*{Output}
Output \texttt{``haiku''} if the given poem is a haiku, and \texttt{``come back next year''} otherwise (quotes for clarity).

\section*{Explanation of sample 1}
One possible decomposition into a haiku is:
\begin{verbatim}
spe-lling ve-ry hard
ear-ly in mor-ning ti-red
i need cov-fe-fe
\end{verbatim}

\section*{Explanation of sample 3}
No matter how the third line is decomposed, it contains 8 syllables instead of 5, so it can not be a haiku.
